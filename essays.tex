\documentclass{article}
\usepackage[UTF8, scheme=plain]{ctex}
\usepackage{amsfonts}
\usepackage{amsmath}
\usepackage{amsthm}
\usepackage{hyperref}
\usepackage[margin=1in]{geometry}
\usepackage{indentfirst}
\linespread{1.2}
\newtheorem{theorem}{定理}
\newtheorem{definition}{定义}
\hypersetup{hidelinks,
         colorlinks=true,
         allcolors=black,
         pdfstartview=Fit,
         breaklinks=true}
\author{李文韬}


\begin{document}
\title{随笔}
\maketitle
\tableofcontents
\newpage


\section{是谁需要文化节}
    学校层面的文化节不仅不利于在校学生开展文化生活,
    而且已经在日趋成熟的精心设计中成为向下规训学生、向上邀功请赏的手段了。\\
    
    首先,我们考虑学生文化节的几个特征。\\
    
    第一,校园文化节采取的形式常常是:
    首先举办开幕式,在“文化节”期间则举办艺术创作和表演比赛,
    择优进入闭幕式暨总结汇演,作为“圆满结束”。\\
    
    在这种“比赛—汇演”模式中,值得注意的有这样一些要素:
    首先,对于某特定艺术形式的创作或表演,因为有评分、排名的要求,
    必须引入一系列评价标准和形制限制,如器乐表演规定时长、绘画作品限制尺幅等。
    第二,汇演中不可能包含所有参与文化节的项目,因此在项目之间存在着竞争关系。\\

    第二,校园文化节作为一种“节日”,自然具备“节日”的特征。
    例如,节日的时间有规律,而且是人为规定;
    文化节在多数学校也是规律性地一年一度,具体的举办时间或由校方直接指定,
    或由“学生自治”性质的团委、学生会“协商决定”。
    又例如,凡是节日都对原本连续着的时间轴进行了一个分划,
    使之成为“节日期间”和“节日以外”两部分,
    节日期间的时间由于节日的存在,具备节日以外的时间所没有的某些性质。
    国庆节的存在是为了庆祝国家政权建立,
    在国庆“节日期间”,国庆节的存在使街道上飘扬红旗、收音机高唱革命歌曲。
    类似地,在文化节的“节日期间”,校园内文化生活呈现出百花齐放的“盛况”,
    而在文化节“节日以外”,公共领域文化生活似乎暂停了,舞台拆毁,观众席重新开始积灰。\\

    对以上这些特征,可以从两个角度分析。
    第一个角度,考虑“比赛—汇演”模式对文化生活的压抑。
    第二个角度,考虑“造节”这一形式考虑文化节对学生和校方分别的意义。\\

    先从第一个角度看。之前已经阐明,“比赛—汇演”模式引入了确定的评价标准和形制限制。\\

    首先考虑评价标准。
    不论评价标准自身有多么包容、完善,这种完善仍然不能改变一个事实:
    参与文化节的过程,已经从学生通过文化表演和创作完成自我的过程,
    变为学生在指导标准之下完成任务的过程。
    当然,存在这样一些个体,他们参与文化节时仍然保持自我完成的态度,不去理会所谓标准。
    然而,如果承认“学生个体完全也可以不受校方影响、独立发展”这样的假设,其推论将是危险的,
    而且将使得对于校园层面的文化节的一切讨论失去意义,
    因此在这里和接下来的讨论中我们暂不承认这个假设。\\

    其次考虑形制限制。
    内容得以传递,不能没有与之符合的形式,艺术内容的传递尤甚。
    因此,对形制的限制直接造成了对内容的限制。
    例如,时长限制使得器乐、声乐类表演不得不勉强割爱、顾此失彼;
    尺幅限制使得视觉类作品本来的意蕴被尺幅本身和观看的过程扭曲。\\

    因此,文化节的比赛模式成功打压了文化活动的多样性,很大程度上剥夺了创作者的主体权利,
    迫使复杂的艺术呈现变得单维化。
    用比喻的方式描述,“比赛—汇演”模式像是
    掌握着汇演名单的公共权力对文化节参与者所收缴的“税”,
    这种税不是钱财,而是权力。
    用另一种比喻的方式描述,若参与者意图进入汇演,
    公共权力便以“放弃创作主动性”为价码,向参与者售出汇演名额。
    然而,与一般的商品交易不同,公共权力通过各种手段,
    使它的顾客们不能意识到自己所付出的价钱;
    相应地,这精明的商人就能避免给出任何确定的质保承诺。
    同学们参与文化节,本来就是为了“飞扬青春”,
    因此,当他们遵照规范化标准进行了创作以后没有得到公共权力的青睐,
    又怎么会产生无端的怨念呢?\\

    最后考虑竞争关系。“比赛—汇演”模式将创作者的一大部分精力从艺术本身转移到竞争之中。
    文化节的参与者在准备过程中,不得不时刻提醒自己要超过其他创作者。
    或许艺术本身的魅力能够使一部分参与者沉浸其中、暂时忘却竞争,
    但他们不可能永远沉浸其中。
    而当他们从纯粹的艺术中抽身、进入自觉的竞争状态,或许就又会感到压力和厌恶了。
    因此,精力的这种转移,无疑是对参与者的艺术追求过程构成妨碍的。\\

    “比赛—汇演”模式对参与者的困扰,就在于它侵蚀了艺术创作的私人领域,
    使参与者在准备和最终呈现过程中时刻有一种“能否超过别人”的顾虑和“被比较的恐惧”。
    文艺创作和表达,本来是逃离公共领域的竞争和外加责任、回归内心的一种自我救赎。
    就连这种来之不易的救赎,也已经被整齐的、大规模的评价、竞争和比较
    转变为一种针对每个文化节参与者的隐形暴力。
    这种暴力将所有个体强行从私人领域押送到公共领域接受监视,
    用凝视的目光不断地抽打恐惧着公共领域的强光的参与者们,
    直到他们从疼痛变得麻木,
    逐渐忘记自己还拥有退回那个稍暗一些的角落——也就是私人领域——的权力。\\

    用一个类比或许能更好地说明问题。
    全景敞视监狱里的犯人们不得不时刻保持警惕、用公共标准约束自己,
    因为他们知道,自己随时都有可能被瞭望塔里的狱警监视着。
    而由于守卫塔的玻璃是单面可视的,狱警能看见犯人,犯人却不能看见狱警,
    因此这种“做坏事被发现的可能”就构成了足够的威慑,
    以至于这种特定结构的监狱只需很少的守卫就可以看管上百名犯人。
    类似地,虽然没有规定文化节参与者若要排练必须有他人在场,
    但参与者们不得不保持竞争的自觉,
    因为他们知道自己的排练行为最终是为竞争服务。
    时刻凝视着参与者的并非评委,而恰恰是参与者们之间的竞争关系本身。
    参与者“看不见”竞争关系,这是因为他不可能知晓其他所有参与者的艺术表达能力;
    即使他能知晓,他也不可能知晓评委将会对其他参与者给出怎样的评价。
    换言之,遮住“狱警”的不是别的,恰恰是“人不能预见未来”这个事实。
    既然人不能预见未来,“落选的可能”就具有足够的威慑力,
    使得仅仅是建立竞争关系本身就能用这种关系对所有参与者进行控制。
    当参与者主动将自己的艺术表达置于文化节“比赛—汇演”体制的支配下,
    他也就自愿开始了对自己的监视,
    从报名的一刻开始,直到落选或汇演结束为止。\\

    以下从第二个角度,即“造节”的角度,考虑文化节对学生和校方各有什么意义。\\

    在探究这个问题之前,我们引入“文化生活”的概念。
    文化生活采取的形式是多样的,但这些形式作为载体,
    其蕴含的内容都是对个人内心或外在世界的反映。
    无论个人内心活动还是外在事物,它们在时间轴上的存在都是连续的。
    文化生活的加工对象时刻存在,但文化内容的表达并非时刻不断,
    这是因为加工对象的积累需要时间
    (某种情感可能再首次出现的一段时间之后才能被清晰地捕捉、表达),
    也是因为加工过程本身需要投入精力
    (基础教育阶段的学生们并不常有闲暇时间可供思考和表达。)
    然而,这种表达上的不连续性因人而异,
    单个学生在一个月内不进行文化表达或许算得上正常,
    但当人口基数扩大到整个年级、整个学校,
    我们很难设想为何他们会鬼使神差地恰巧在一年的某几个月中集体静默,
    而在另外的一个月后又一齐产出了文化内容。
    另外,即使同样不产生内容,
    “有思考但仍然处于加工阶段”的静默和“疲于应对生活,完全没有精力思考,
    更不可能表达”的静默又不相同。\\

    总之,“文化生活”描述的是一种积极主动的思考状态在文艺形式中的表现,
    它是一种“生活”,即一种常态化、持续性的思考和表达。
    文化生活允许主动的静默,但厌恶被动的、疲于应付的静默;
    文化生活中的产出时刻因人而异,
    设想在某个人数成百上千的集体中出现文化产出行为在时间上的绝对一致性是荒谬的。\\

    \newpage

    从文化生活本身的性质可以看出,它不应该被限制在某个特定的时间段内。
    沉默太久的人,会忘记如何歌唱。
    同样,当一个学生长期处在与文化生活隔绝的状态中,
    或许突然的“解严”所引发的热情能暂时支撑他们进行思考和表达,
    但他们本身技艺的生疏和感官的麻木也将给他们的这种表达行为造成很大困难。
    基础教育阶段,学生的行为规范和课业进度都受严格控制,
    在这样的生活里还能有余下的时间进行思考和表达已经不容易,
    现在却要求学生不仅要会表达、敢表达,
    而且要在学业与文化生活之间切换模式,恐怕太难为他们了。\\

    事实上,文化节所造成的困境也在于此。
    拥有文化生活的学生,无论是文化节“期间”还是文化节“之外”都进行着文化生活,
    参与文化节也只是从他们平时的积累中选择合适的部分向外输出;
    至于另一部分——常常是一大部分——学生,
    在文化节以外几乎没有文化生活(当然,文化节期间也不一定有),
    他们中参与文化节的往往是出于集体责任感“被参与”、“被报名”了。
    对于他们,文化节只是学习生活中又一件麻烦事。
    在这部分同学看来,拥有了文化生活的同学们都称得上“某某大佬”,
    不论这个“某某”是哪一种艺术形式,都与他们无关。
    文化节充当了一个屏障,将以上这两部分学生分离开,
    甚至将他们对立起来:比赛、竞争、淘汰、汇演,
    文化节可以说是在观众和艺术表达之间建立了屏障,
    让文艺活动看似光鲜亮丽,实则高不可攀,
    用高耸、遥远的舞台托起了文化生活,使其失去群众基础,
    从而“捧杀”了文化生活。
    这种捧杀并非一般的先提高其地位再降低之,而是无节制的提高地位和参与标准,
    以至于学生们认为参与这种活动是高标准严要求、费时费力、过分奢侈、力所不能及的,
    从而切断文化生活的源头活水。\\

    除了捧杀,文化节还有着“分而治之”的功效。
    如果学生有文化生活,就利用其表达欲,将其表达过程纳入庞大的规训机器中,
    这机器的原理和效果,之前已经阐明;
    如果学生没有文化生活,就进行隐形的劝退,
    表面上以班级为单位强制参与,
    实则用“比赛—汇演”这种高压力模式让学生切身体会到长久失语后被迫张口的惶惑,
    起到遏制文化生活向更广大学生群体传播的作用。\\

    因此,从学生本身的角度看,文化节为文化造了一个节,
    这个所谓节日恰恰用钝器慢慢杀死了学生中间的文化的魂灵。\\

    现在从学校角度考虑文化节的意义。
    首先,文化节起到了控制学生的作用。
    替代了常态化的自由表达的,是定时定量的命题创作,
    不仅剥夺了学生的主动性,而且削弱了文化节以外进行文化生活的正义性
    (同学们,不要再唱歌,不要再跳舞了,要知道,你们可是拥有文化节的!)
    和可行性(闭幕式后两礼拜要期末考了,前面排练时好像还欠了作业…)。
    这样,就起到了类似“疏堵结合”的作用。
    其次,学校文化节给学生和社团所进行的文化活动打上集体的烙印,
    强化了“学校”这个概念在文化生活中的存在,淡化了具体的个人。
    这样,文化节作为一种集中的统一活动,就为学校层面对外宣传营造了有利条件,
    不仅方便统一的拍摄、汇总和报道,还将全年的文化生活浓缩在短时间内呈现,
    活动浓度升高,宣传也就能更加密集;宣传密集了,宣传的效果也“更好”。\\

    不平衡和不充分从来不是发展的障碍,虚伪才是。
    对于初高中生,文化生活或许确实是奢侈的,
    但校方难道不该因此而更加注意避免拔苗助长、避免搞文化大跃进吗?
    也许现在的情况仍然算是乐观,
    但如果从不向文化节这样的活动中注入一点实事求是、脚踏实地的态度,
    长此以往,恐怕我们就能建成世界一流有文化高中了。\\

    \newpage
    
\section{谁没有理想}


\end{document}